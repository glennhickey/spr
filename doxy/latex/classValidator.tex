\section{Validator$<$ T $>$ Class Template Reference}
\label{classValidator}\index{Validator@{Validator}}
Validator class is responsible for verifying that a tree is properly formed.  


{\tt \#include $<$validator.h$>$}

\subsection*{Public Member Functions}
\begin{CompactItemize}
\item 
{\bf Validator} (int n)
\item 
{\bf $\sim$Validator} ()
\item 
bool {\bf valid\-Tree} (const {\bf UPTree}$<$ T $>$ \&)
\begin{CompactList}\small\item\em Ensure that the tree is composed of a 3 valid binary subtrees. \item\end{CompactList}\item 
int {\bf subtree\-Size} (const {\bf UPTree}$<$ T $>$ \&, int)
\begin{CompactList}\small\item\em returns size of subtree in number of tokens if valid. \item\end{CompactList}\item 
int {\bf reorder} (const {\bf UPTree}$<$ T $>$ \&, {\bf UPTree}$<$ T $>$ \&)
\begin{CompactList}\small\item\em Ensure that for each internal node of the tree, the minimum leaf label in its left subtree is smaller than the minimum leaf label in its right subtree. \item\end{CompactList}\item 
int {\bf retrifurcate} (const {\bf UPTree}$<$ T $>$ \&, {\bf UPTree}$<$ T $>$ \&, T)
\begin{CompactList}\small\item\em Retrifurcation of an unrooted tree is analagous to rerooting a rooted tree. \item\end{CompactList}\item 
const {\bf Bracket\-Table}$<$ T $>$ \& {\bf get\-BTable} () const 
\begin{CompactList}\small\item\em Return a reference to the bracket table as it is sometimes reused in the kernelization. \item\end{CompactList}\item 
int {\bf get\-Size} () const 
\begin{CompactList}\small\item\em Get the size of the bracket table. \item\end{CompactList}\end{CompactItemize}
\subsection*{Protected Member Functions}
\begin{CompactItemize}
\item 
int {\bf reorder\-Subtree} (const {\bf UPTree}$<$ T $>$ \&, int, {\bf UPTree}$<$ T $>$ \&, int)
\begin{CompactList}\small\item\em Reorder a subtree to ensure that at each internal node, the minimum leaf label in the left subtree is smaller than the minimum leaf label in the right subtree. \item\end{CompactList}\end{CompactItemize}
\subsection*{Protected Attributes}
\begin{CompactItemize}
\item 
{\bf Bracket\-Table}$<$ T $>$ {\bf \_\-table}
\begin{CompactList}\small\item\em Table of matching brackets used to store each bifurcation for quick lookup. \item\end{CompactList}\item 
std::stack$<$ T $>$ {\bf \_\-stack}
\begin{CompactList}\small\item\em General purpose stack used mostly for counting brackets and tree traversals. \item\end{CompactList}\end{CompactItemize}
\subsection*{Private Member Functions}
\begin{CompactItemize}
\item 
{\bf Validator} ()
\end{CompactItemize}


\subsection{Detailed Description}
\subsubsection*{template$<$class T$>$ class Validator$<$ T $>$}

Validator class is responsible for verifying that a tree is properly formed. 

It checks that it is binary, unrooted, and that all the brackets match. Also provides functionality to ensure that the tree is ordered consistently to provide a unique representation for each unique topology.

\begin{Desc}
\item[Author:]Glenn Hickey \end{Desc}




\subsection{Constructor \& Destructor Documentation}
\index{Validator@{Validator}!Validator@{Validator}}
\index{Validator@{Validator}!Validator@{Validator}}
\subsubsection{\setlength{\rightskip}{0pt plus 5cm}template$<$class T$>$ {\bf Validator}$<$ T $>$::{\bf Validator} (int {\em size})}\label{classValidator_a0}


\begin{Desc}
\item[Parameters:]
\begin{description}
\item[{\em size}]Size of table. Must be large enough to contain each token of the tree being processed. \end{description}
\end{Desc}
\index{Validator@{Validator}!~Validator@{$\sim$Validator}}
\index{~Validator@{$\sim$Validator}!Validator@{Validator}}
\subsubsection{\setlength{\rightskip}{0pt plus 5cm}template$<$class T$>$ {\bf Validator}$<$ T $>$::$\sim${\bf Validator} ()}\label{classValidator_a1}


\index{Validator@{Validator}!Validator@{Validator}}
\index{Validator@{Validator}!Validator@{Validator}}
\subsubsection{\setlength{\rightskip}{0pt plus 5cm}template$<$class T$>$ {\bf Validator}$<$ T $>$::{\bf Validator} ()\hspace{0.3cm}{\tt  [private]}}\label{classValidator_d0}




\subsection{Member Function Documentation}
\index{Validator@{Validator}!getBTable@{getBTable}}
\index{getBTable@{getBTable}!Validator@{Validator}}
\subsubsection{\setlength{\rightskip}{0pt plus 5cm}template$<$class T$>$ const {\bf Bracket\-Table}$<$ T $>$ \& {\bf Validator}$<$ T $>$::get\-BTable () const\hspace{0.3cm}{\tt  [inline]}}\label{classValidator_a6}


Return a reference to the bracket table as it is sometimes reused in the kernelization. 

\index{Validator@{Validator}!getSize@{getSize}}
\index{getSize@{getSize}!Validator@{Validator}}
\subsubsection{\setlength{\rightskip}{0pt plus 5cm}template$<$class T$>$ int {\bf Validator}$<$ T $>$::get\-Size () const\hspace{0.3cm}{\tt  [inline]}}\label{classValidator_a7}


Get the size of the bracket table. 

Corresponds to the maximum size of a tree that can be validated. \index{Validator@{Validator}!reorder@{reorder}}
\index{reorder@{reorder}!Validator@{Validator}}
\subsubsection{\setlength{\rightskip}{0pt plus 5cm}template$<$class T$>$ int {\bf Validator}$<$ T $>$::reorder (const {\bf UPTree}$<$ T $>$ \& {\em itree}, {\bf UPTree}$<$ T $>$ \& {\em otree})}\label{classValidator_a4}


Ensure that for each internal node of the tree, the minimum leaf label in its left subtree is smaller than the minimum leaf label in its right subtree. 

\begin{Desc}
\item[Parameters:]
\begin{description}
\item[{\em itree}]Tree to reorder \item[{\em otree}]Tree to which the reordered result is written \end{description}
\end{Desc}
\begin{Desc}
\item[Returns:]Number of subtrees that had to be reordered. \end{Desc}
\index{Validator@{Validator}!reorderSubtree@{reorderSubtree}}
\index{reorderSubtree@{reorderSubtree}!Validator@{Validator}}
\subsubsection{\setlength{\rightskip}{0pt plus 5cm}template$<$class T$>$ int {\bf Validator}$<$ T $>$::reorder\-Subtree (const {\bf UPTree}$<$ T $>$ \& {\em itree}, int {\em pos}, {\bf UPTree}$<$ T $>$ \& {\em otree}, int {\em opos})\hspace{0.3cm}{\tt  [protected]}}\label{classValidator_b0}


Reorder a subtree to ensure that at each internal node, the minimum leaf label in the left subtree is smaller than the minimum leaf label in the right subtree. 

Ordering all subtrees in this fashion is essential for a unique representation required by the cache. This is done in a single scan. \begin{Desc}
\item[Parameters:]
\begin{description}
\item[{\em itree}]Input tree. \item[{\em pos}]Starting position of subtree in input tree to reorder. \item[{\em otree}]Output tree. \item[{\em opos}]Starting position of subtree in output tree. \end{description}
\end{Desc}
\begin{Desc}
\item[Returns:]Number of subtrees that were out of order. \end{Desc}
\index{Validator@{Validator}!retrifurcate@{retrifurcate}}
\index{retrifurcate@{retrifurcate}!Validator@{Validator}}
\subsubsection{\setlength{\rightskip}{0pt plus 5cm}template$<$class T$>$ int {\bf Validator}$<$ T $>$::retrifurcate (const {\bf UPTree}$<$ T $>$ \& {\em itree}, {\bf UPTree}$<$ T $>$ \& {\em otree}, T {\em root})}\label{classValidator_a5}


Retrifurcation of an unrooted tree is analagous to rerooting a rooted tree. 

In order to ensure a unique representation, we enforce that the leaf with label=0 correspond to the first subtree of the trifurcation. The two subtrees rooted at leaf 0's neighbours will correspond to the second and third trifurcations.

Example: Input: ((4,5),((0,3),2),(1,6)) Output: (0,3,(((4,5),2),(6,1)))

This is a rather involved routine that will hopefuly be fully described somewhere in the thesis.

Note that it must be used in conjunction (ie before) reorder to generate a unique representation \begin{Desc}
\item[Parameters:]
\begin{description}
\item[{\em itree}]Input tree \item[{\em otree}]Tree in which retrifurcated itree will be copied \item[{\em root}]Label of 'root' that will be set as first trifurcation. I use 0 \end{description}
\end{Desc}
\begin{Desc}
\item[Returns:]0 if input is already retrifurcated or 1 if work had to be done \end{Desc}
\index{Validator@{Validator}!subtreeSize@{subtreeSize}}
\index{subtreeSize@{subtreeSize}!Validator@{Validator}}
\subsubsection{\setlength{\rightskip}{0pt plus 5cm}template$<$class T$>$ int {\bf Validator}$<$ T $>$::subtree\-Size (const {\bf UPTree}$<$ T $>$ \& {\em tree}, int {\em pos})}\label{classValidator_a3}


returns size of subtree in number of tokens if valid. 

0 otherwise Checks if it starts with a (. Checks that all brackets match. Checks that it is strictly bifurcating. \begin{Desc}
\item[Parameters:]
\begin{description}
\item[{\em tree}]Input tree \item[{\em pos}]Starting position of subtree to analyze \end{description}
\end{Desc}
\index{Validator@{Validator}!validTree@{validTree}}
\index{validTree@{validTree}!Validator@{Validator}}
\subsubsection{\setlength{\rightskip}{0pt plus 5cm}template$<$class T$>$ bool {\bf Validator}$<$ T $>$::valid\-Tree (const {\bf UPTree}$<$ T $>$ \& {\em tree})}\label{classValidator_a2}


Ensure that the tree is composed of a 3 valid binary subtrees. 

Useful mostly for testing input. 

\subsection{Member Data Documentation}
\index{Validator@{Validator}!_stack@{\_\-stack}}
\index{_stack@{\_\-stack}!Validator@{Validator}}
\subsubsection{\setlength{\rightskip}{0pt plus 5cm}template$<$class T$>$ std::stack$<$T$>$ {\bf Validator}$<$ T $>$::{\bf \_\-stack}\hspace{0.3cm}{\tt  [protected]}}\label{classValidator_p1}


General purpose stack used mostly for counting brackets and tree traversals. 

\index{Validator@{Validator}!_table@{\_\-table}}
\index{_table@{\_\-table}!Validator@{Validator}}
\subsubsection{\setlength{\rightskip}{0pt plus 5cm}template$<$class T$>$ {\bf Bracket\-Table}$<$T$>$ {\bf Validator}$<$ T $>$::{\bf \_\-table}\hspace{0.3cm}{\tt  [protected]}}\label{classValidator_p0}


Table of matching brackets used to store each bifurcation for quick lookup. 



The documentation for this class was generated from the following files:\begin{CompactItemize}
\item 
{\bf validator.h}\item 
{\bf validator\_\-impl.h}\end{CompactItemize}
